экипаж, наводивший недоумение насчет своего названия
 Он не был похож ни на тарантас, ни на коляску, ни на бричку, а был скорее похож на толстощекий выпуклый арбуз, поставленный на колеса
 Щеки этого арбуза, то есть дверцы, носившие следы желтой краски, затворялись очень плохо по причине плохого состояния ручек и замков, кое-как связанных веревками
 Арбуз был наполнен ситцевыми подушками в виде кисетов, валиков и просто подушек, напичкан мешками с хлебами, калачами, кокурками, скородумками и кренделями из заварного теста
 Пирог-курник и пирог-рассольник выглядывали даже наверх
 Запятки были заняты лицом лакейского происхожденья, в куртке из домашней пеструшки, с небритой бородою, подернутою легкой проседью, — лицо, известное под именем «малого»
 Шум и визг от железных скобок и ржавых винтов разбудили на другом конце города булочника, который, подняв свою алебарду, закричал спросонья что стало мочи: «Кто идет?» — но, увидев, что никто не шел, а слышалось только издали дребезжанье, поймал у себя на воротнике какого-то зверя и, подошед к фонарю, казнил его тут же у себя на ногте
 После чего, отставивши алебарду, опять заснул по уставам своего рыцарства
 Лошади то и дело падали на передние коленки, потому что не были подкованы, и притом, как видно, покойная городская мостовая была им мало знакома
 Колымага, сделавши несколько поворотов из улицы в улицу, наконец поворотила в темный переулок мимо небольшой приходской церкви Николы на Недотычках и остановилась пред воротами дома протопопши
 Из брички вылезла девка, с платком на голове, в телогрейке, и хватила обоими кулаками в ворота так сильно, хоть бы и мужчине (малый в куртке из пеструшки был уже потом стащен за ноги, ибо спал мертвецки)
 Собаки залаяли, и ворота, разинувшись наконец, проглотили, хотя с большим трудом, это неуклюжее дорожное произведение
 Экипаж въехал в тесный двор, заваленный дровами, курятниками и всякими клетухами; из экипажа вылезла барыня: эта барыня была помещица, коллежская секретарша Коробочка
 Старушка вскоре после отъезда нашего героя в такое пришла беспокойство насчет могущего произойти со стороны его обмана, что, не поспавши три ночи сряду, решилась ехать в город, несмотря на то что лошади не были подкованы, и там узнать наверно, почем ходят мертвые души и уж не промахнулась ли она, боже сохрани, продав их, может быть, втридешева
 Какое произвело следствие это прибытие, читатель может узнать из одного разговора, который произошел между одними двумя дамами
 Разговор сей… но пусть лучше сей разговор будет в следующей главе
